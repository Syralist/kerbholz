\documentclass[aspectratio=1610,svgnames]{beamer}

\usepackage{lmodern}
\usepackage[T1]{fontenc}
\usepackage[ngerman]{babel}
\usepackage{selinput}
\SelectInputMappings{%
   adieresis={ä},
   germandbls={ß}
   }
\usepackage{csquotes}

\usetheme{PaloAlto}  %% Themenwahl

\setbeamercovered{transparent}
%\setbeamertemplate{footline}[frame number]
\usecolortheme{spruce}		% grün
\usecolortheme[named=MSUgreen]{structure}
\setbeamercolor{logo}{bg=white}
\setbeamercolor{title in sidebar}{fg=white}
\newcommand{\hshblogo}{\includegraphics[width=1.3cm]{space-logo}}
\newcommand{\divider}[1]{\begin{frame} %
\begin{alertblock}{} %
\centering\usebeamerfont{section title}#1 %
\end{alertblock} %
\end{frame}}
 
\newcommand{\code}[1]{\texttt{#1}}

\title{Ein digitales Kerbholz mit Python und Flask}
\author{Thomas Helmke}
\date{when its done}
\logo{\includegraphics[width=1.1cm]{space-logo}}
 
\begin{document}
\maketitle
\frame{\tableofcontents}

\section{Einleitung}
\divider{\insertsection}
\begin{frame}%[<+->] %%Eine Folie
	\frametitle{Worum geht es?} %%Folientitel
	\begin{itemize}
        \item eine Einführung in Web-Apps mit Python und Flask
	\end{itemize}
\end{frame}

% \section{Grundlagen}
% \divider{\insertsection}
% \subsection{Basiskörper}
% \divider{\insertsubsection}
% \begin{frame}[<+->]
%     \frametitle{Basiskörper}
%     \begin{itemize}
%         \item Würfel -- \code{cube([xLänge, yLänge, zLänge]);}
%         \item Zylinder -- \code{cylinder(h = Höhe, r1 = RadiusUnten, r2 = RadiusOben);}
%         \item Kugel -- \code{sphere(r = Radius);}
%     \end{itemize}
% \end{frame} 
% \begin{frame}[<+->]
%     \frametitle{optionale Parameter}
%     \begin{itemize}
%         \item \code{center} -- zentriert den Körper im Koordinatenursprung
%         \item \code{\$fn} -- Anzahl der Flächen für runde Körper
%         \item \code{d} -- Durchmesser statt Radius
%     \end{itemize}
% \end{frame} 
% \subsection{Basisoperationen}
% \divider{\insertsubsection}
% \begin{frame}[<+->]
%     \frametitle{Basisoperationen I}
%     \begin{itemize}
%         \item Verschieben -- \code{translate([x, y, z])}
%         \item Rotieren -- \code{rotate([xGrad, yGrad, zGrad])}
%         \item Skalieren -- \code{scale([xFaktor, yFaktor, zFaktor])}
%         \item Spiegeln -- \code{mirror([x,y,z])}
%     \end{itemize}
% \end{frame} 
% \begin{frame}[<+->]
%     \frametitle{Beispiel}
%     \code{translate([2,3,4])\newline%
%         \hspace*{1em}rotate([45,0,0])\newline%
%         \hspace*{2em}scale([0.5,1,2])\newline%
%         \hspace*{3em}cube([1,1,1]);}
% \end{frame} 
% \begin{frame}[<+->]
%     \frametitle{Basisoperationen II}
%     \begin{itemize}
%         \item Vereinen -- \code{union()\{\}}
%         \item Abziehen -- \code{difference()\{\}}
%         \item Überschneidung -- \code{intersection()\{\}}
%         % \item Spiegeln -- \code{mirror([x,y,z])}
%     \end{itemize}
% \end{frame} 

% \section{Fortgeschrittene Techniken}
% \divider{\insertsection}
% \subsection{Variablen}
% \divider{\insertsubsection}
% \begin{frame}[<+->]
%     \frametitle{Variablen I}
%     \begin{itemize}
%         \item Werte können in Variablen gespeichert werden
%         \item Variablen können mathematisch verrechnet werden
%     \end{itemize}
%     \uncover<+->{%
%     \code{xsize = 3.5;\newline%
%         ysize = xsize;\newline%
%         zsize = 0.5*xsize;}}
% \end{frame}
% \begin{frame}[<+->]
%     \frametitle{Variablen II}
%     \begin{itemize}
%         \item Variablen können an Funktionen übergeben werden
%     \end{itemize}
%     \uncover<+->{%
%     \code{translate(v=[xpos*(gap+xsize)+0.5*xsize,ypos*(gap+ysize)+0.5*ysize,0.5*zsize])\newline%
%         \hspace*{1em}cylinder(h=1.1*zsize, r=0.4*xsize, \$fn=20, center=true);}}
% \end{frame}
% \subsection{Iteration}
% \divider{\insertsubsection}
% \begin{frame}[<+->]
%     \frametitle{Iteration}
%     \begin{itemize}
%         \item For-Schleife, wie aus anderen Sprachen bekannt
%         \item Entweder Bereich vorgeben oder Vektor von Elementen
%         \item \code{for(variable = [start : increment : end])}
%         \item \code{for(variable = [vector])}
%     \end{itemize}
% \end{frame}
% \subsection{Module}
% \divider{\insertsubsection}
% \begin{frame}[<+->]
%     \frametitle{Module I}
%     \begin{itemize}
%         \item beliebige Funktionen zu Modulen zusammenfassen
%         \item Wiederverwendbarkeit von Code
%         \item Module können in externe Dateien ausgelagert werden
%     \end{itemize}
% \end{frame}
% \begin{frame}[<+->]
%     \frametitle{Module II}
%     Datei \enquote{mycube.scad}\newline
%     \code{module mycube(var1, var2)\newline%
%         \{\newline%
%         \hspace*{1em}cube([var1,var2,var1+var2]);\newline%
%         \}\newline%
%         mycube(1,2);}
% \end{frame}
% \begin{frame}[<+->]
%     \frametitle{Module III}
%     Datei \enquote{main.scad}\newline
%     \code{use <mycube.scad>;\newline%
%         mycube(2,3);\newline%
%         translate([1,2,3])\newline%
%         \hspace*{1em}mycube(1,2);}
% \end{frame}


\section{Weitere Infos}
\divider{\insertsection}
\begin{frame}
    \frametitle{Weitere Infos}
    \begin{itemize}
        \item \url{https://github.com/Syralist/kerbholz}
        \item \url{https://www.sqlalchemy.org/}
        \item \url{http://flask-sqlalchemy.pocoo.org/}
        % \item \url{https://en.wikibooks.org/wiki/OpenSCAD_User_Manual}
    \end{itemize}
\end{frame}
\end{document}
